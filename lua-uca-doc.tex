\documentclass{ltxdoc}


\usepackage[english]{babel}
\usepackage{hyperref}
\newcommand\authormail[1]{\footnote{\textless\url{#1}\textgreater}}
\ifdefined\HCode
\renewcommand\authormail[1]{\space\textless\Link[#1]{}{}#1\EndLink\textgreater}
\fi

\usepackage{fontspec}
\setmainfont{TeX Gyre Schola}
% \setmonofont[Scale=MatchLowercase]{Inconsolatazi4}
\IfFontExistsTF{Noto Sans Mono Regular}{%
  \setmonofont[Scale=MatchLowercase]{Noto Sans Mono Regular}
}{\setmonofont{NotoMono-Regular.ttf}}

\usepackage{upquote}
\usepackage{microtype}
\usepackage[hybrid]{markdown}
\usepackage{luacode}

\title{The \texttt{Lua-UCA} library}
\author{Michal Hoftich\authormail{michal.h21@gmail.com}}
\date{Version \version\\\gitdate}
\begin{document}
\maketitle
\tableofcontents
\markdownInput{README.md}

\section{Available languages}

The \texttt{lua-uca-languages} library provides the following langauges:
\bgroup\ttfamily
\begin{luacode*}
-- get list of the currently supported languages directly from the library
local l = {}
local languages = require "lua-uca-languages"
for lang, _ in pairs(languages) do
l[#l+1] = lang:gsub("_", '\\_')
end

table.sort(l)
tex.print(table.concat(l, ", "))
\end{luacode*}
\egroup

\markdownInput{CHANGELOG.md}

\end{document}
